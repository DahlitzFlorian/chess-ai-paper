% !TeX root = ../../paper.tex
\subsubsection{Transpositionstabellen} \label{ch:transposition-tables}

Im Schach treten viele \textit{Transpositionen} auf.
Transpositionen sind verschiedene Permutationen einer Zugsequenz, die in die selbe Stellung münden [\cite{Russell2010}].
Folgendes Beispiel soll dies näher illustrieren: Weiß hat einen Zug, $a1$, der von Schwarz mit $b1$ beantwortet werden kann, und einen nicht verwandten Zug $a2$ auf der anderen Seite des Bretts, der mit $b2$ beantwortet werden kann.
Beide Sequenzen $[a1, b1, a2, b2]$ und $[a2, b2, a1, b1]$ münden in der gleichen Stellung.
Es ist vorteilhaft, das Ergebnis der ersten Zugsequenz in einer Hash Tabelle abzulegen und beim zweiten Mal das Ergebnis aus der Hash Tabelle zu verwenden, anstatt es von Neuem zu berechnen.
Das Verwenden von Transpositionstabellen kann einen signifikanten Einfluss haben.
So kann die Verwendung dafür sorgen, dass in einigen Fällen der Schachcomputer bis zu der doppelten Tiefe suchen kann [\cite{Russell2010}].
