% !TeX root = ../../paper.tex
\subsection{Tools}

Zur Umsetzung der algorithmischen Anteile werden ausschließlich Funktionalitäten der Python Standardbibliothek verwendet.
Für die Repräsentation eines Schachspiels sowie die Visualisierung wird das Paket \textit{Python-Chess} verwendet.
Das Paket ist im \ac{PyPI} verfügbar.
Aus diesem Grund kann es, wie in Listing~\ref{lst:tools_python-chess_pip-installation} dargestellt, über den Python Package Installer \textit{pip} installiert werden.

\bigskip

\begin{lstlisting}[caption={Installation des Python-Chess Pakets via pip}, captionpos=b, label={lst:tools_python-chess_pip-installation}, language=sh]
$ python -m pip install python-chess
\end{lstlisting}

\noindent Des Weiteren wird zur Dokumentation der Umsetzung und für das Spielen gegen den Schachcomputer ein Jupyter Notebook verwendet.
